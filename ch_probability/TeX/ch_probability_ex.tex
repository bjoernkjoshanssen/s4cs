


%_______________
\section{Exercises}
%_______________
%\subsection{Defining probability}

% 1

% 2
% 3
% 5
%\textC{\pagebreak}

%_______________
%\subsection{Conditional probability}

% 15


% 16
% 17
%\newpage
%_______________
%\subsection{Sampling from a small population}
%_______________
%\subsection{Random variables}
%\textC{\newpage}
% 38
%\textC{\pagebreak}

%_______________
%\subsection{Continuous distributions}

\eoce{\qt{Normal distributions\label{normal_dist}}
        The \textit{normal distribution} is a continuous probability distribution defined
        by the probability density function $\varphi(x) = \frac{1}{\sqrt{2 \pi \sigma^2}}e^{-\frac{(x - \mu)^2}{2 \sigma^2}}$ for fixed constants $\mu$ and $\sigma$.
        \begin{parts}
            \item 
                Use that ${\displaystyle \int_{-\infty}^{\infty} e^{-\frac{x^2}{2}} dx = \sqrt{2 \pi}}$ to show that $\varphi$ is indeed a probability density function.
            \item 
                Show that the mean, median, and mode of the normal distribution are all equal to $\mu$.
            \item
                Show that the variance of the normal distribution is equal to $\sigma^2$ and
                therefore that the standard deviation of the normal distribution is equal to $\sigma$.
        \end{parts}
}{}

\eoce{\qt{Skewed normals\label{skewed}}
        Let $\varphi$ be as in the previous problem with $\mu=0$ and $\sigma=1$ (the ``standard normal'' case).
        Define $\Phi(x) := {\displaystyle \int_{-\infty}^x \varphi(t) dt}$, the \textit{cumulative distribution function of $\varphi$}. Fix $c \in \mathbb R$ and define the \textit{skewed normal distribution with skewness parameter $c$} by 
        \begin{center}
            ${\displaystyle f_c(x) := \frac{\varphi(x) \Phi(cx)}{\Phi(0)}}$
        \end{center}
        Note that $f_c(x) = \varphi(x)$ when $c = 0$.
        \begin{parts}
        \item Show that $\int_{-\infty}^\infty f_c(x)<\infty$.
            \item 
                Show that $f_c$ is indeed a probability density function. (Hint: This requires multivariable calculus. $\int_{-\infty}^\infty f_c(x)\,dx$ is the probability that for two independent standard normal random variables $X$ and $Y$, we have $\P(Y\le cX)$:
                \[
                		\P(Y\le cX) = \int_{-\infty}^\infty \varphi(x)\int_{-\infty}^{cx} \varphi(y)\,dy\,dx.
                \]
                This is $1/2=\Phi(0)$ because $\{(x,y): y\le cx\}$ is a half-plane and the joint pdf of $X$ and $Y$ is spherically symmetric.)
            \item
                Find the mean of $f_c(x)$. %, median, and mode of
                (Hint: the mean is
                \[
                		E(2X [Y\le cX]) = 2 \frac1{2\pi}\int_{-\infty}^\infty \int_{-\infty}^{cx} x e^{-r^2/2} \,dy\,dx.
		\]
                This is $\frac1\pi \int_0^\infty r^2 e^{-r^2/2}\,dr \int_{\tan^{-1}(c)-\pi}^{\tan^{-1}(c)}\cos\theta\,d\theta$ using polar coordinates in multiple integrals.
                We have $\int_{\tan^{-1}(c)-\pi}^{\tan^{-1}(c)}\cos\theta\,d\theta = [\sin\theta]_{\tan^{-1}(c)-\pi}^{\tan^{-1}(c)} = 2\frac{c}{\sqrt{1+c^2}}$
                since if $\tan\theta=c$ then $\sin\theta=c/\sqrt{1+c^2}$ (draw a triangle). On the other hand
                $\int_0^\infty r^2e^{-r^2/2}\,dr = \frac12 \int_{-\infty}^\infty x^2 e^{-x^2/2}\,dx = \frac12 \sqrt{2\pi}$ using the fact that $E(X^2)=1$ for a standard normal $X$.
                %For the median, the equation to solve for $m$ is $\P(X\le m, Y\le cX)=1/2$, but it appears tricky to solve exactly. There is also no analytic expression for the mode.
                )
            \item Show that the mode of $f_c(x)$ is unique (there is no simple analytic expression for it, however).
            %\item
             %   For which $c$ is the mean of $f_c(x)$ greater than the median? Less than the median? Compare your answers to 2c.
             \item Show that $g(x)=\frac{d}{dx}(\Phi(x))^2$ is also a probability density function. (Hint: $\int_{-\infty}^\infty g(x)\,dx = 1^2-0^2=1$.) How does it relate to $f_c$?
        \end{parts}
}{}

% 43
\eoce{\qt{Cauchy distribution\label{cauchy_dist}}
        The \textit{Cauchy distribution} is a continuous probability distribution defined by the probability density function $f(x) = \frac{1}{\pi(1+x^2)}$
        \begin{parts}
            \item 
                Show that
                $f$ is indeed a probability density function.
            \item
                Find the median and the mode of the Cauchy distribution.
            \item
                Show that the mean of the Cauchy distribution does not exist.
            \item
                Conclude that the weak (and therefore the strong) Law of Large Numbers fails for the Cauchy distribution. 
        \end{parts}
}{}

\eoce{\qt{Bayesian statistics}
Suppose $X$ is a Bernoulli random variable with unknown parameter $p$. Your initial assumption is that $p$ is distributed uniformly on $[0,1]$.
Then you observe in sequence the values $X=1,X=0,X=0$. What is your new p.d.f. for $p$?
}{}
