
\section{Exercises}

%\Comment{need to add in conceptual questions about ci for mean, such as 9-12 of original 04e.tex from openintro }

%__________________
%\subsection{One-sample means with the t-distribution}

% 1


\eoce{\qt{The price of pairs}
        As in Section \ref{pairedData}, let $\omega$ be a textbook, let $X(\omega)$ be its price in the UCLA bookstore, and let $Y(\omega)$ be its price on Amazon. Give an example of sets of observations that are paired and an example of sets of observations that are not paired, where an observation is the price of a textbook.
}{}

\eoce{\qt{Tee of infinity}
        Informally describe the quantity $\underset{n \rightarrow \infty}{\text{lim}}\frac{S}{\sqrt{n}}$, where $S$ is the standard error. Conclude that the $t$-distribution approximates the standard distribution as the number of degrees of freedom approaches infinity.
}{}

\eoce{\qt{Abnormality}
Briefly describe why each of the following distributions might fail to be normal.
        \begin{enumerate}
            \item 
                200 homes in a 1000-home district are surveyed about their voting habits.
            \item
                10 homes in a 1000-home district are surveyed about their preference for one of two candidates.
        \end{enumerate}
}{}

\eoce{\qt{Standard error of a difference}
        If $p_1$ and $p_2$ are two population proportions with sample sizes $n_1$ and $n_2$ respectively such that $\hat{p_1}$ and $\hat{p_2}$ are normal, verify directly that $SE_{\hat{p_1}-\hat{p_2}} = \sqrt{\frac{p_1(1-p_1)}{n_1} + \frac{p_2(1-p_2)}{n_2}}$.
}{}

