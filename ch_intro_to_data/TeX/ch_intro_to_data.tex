\chapter{Introduction to Data}
\label{introductionToData}

\section{Continuous and discrete}
The distinction between continuous and discrete data is common in mathematics. 
\emph{Continuous} data take values in the real numbers $\mathbb{R}$. 
\emph{Discrete} data take values in a finite or countably infinite set, such as the natural numbers $\mathbb{N}$.

If salaries are measured in integer multiples of \$100, they can be considered discrete. 
In reality, a salary is always an integer multiple of one cent, but this discreteness is somewhat artificial: nothing prevents us from using salaries involving fractions of a cent except convention.

\begin{example}{Impossible salaries}
	Estelle has \$100{,}000 to spend on salaries for three employees. Since the employees are indistinguishable, other things being equal, she decides to give each of them a salary of
	\[
		\$\frac{100{,}000}{3}.
	\]
\end{example}

\begin{example}{Discrete time?}
	Time is usually modeled using real numbers, although it remains unknown in theoretical physics whether time is actually discrete.\footnote{\url{https://physics.stackexchange.com/questions/35674/is-time-continuous-or-discrete}}
	If time were discrete, there would be a smallest possible time step.
\end{example}

\section{Categorical variables}
Among categorical variables, the \emph{ordinal} ones possess an intrinsic ordering. 
Some variables have structure without being numerical or ordinal. 
For example, consider the variable \var{direction}:
\begin{quote}
	North, Northwest, West, Southwest, South, Southeast, East, Northeast.
\end{quote}
These values are naturally ordered around a circle.

\begin{exercise}
Give another example of a structured variable that is not numerical or ordinal. 
Perhaps another variable that is naturally arranged in a “wheel”?%
\footnote{A common example is color, which can be arranged cyclically: Red, Orange, Yellow, Green, Blue, Purple.}
\end{exercise}

\begin{exercise}
	What kind of mathematical structure can you find in the variable “state of residence” among U.S.\ states?%
	\footnote{For example, considering whether two states share a border gives rise to a graph structure, as studied in discrete mathematics.}
\end{exercise}

\section[Examining numerical data]{Examining numerical data}
\label{numericalData}

\begin{termBox}{\tBoxTitle{Mean}%
The sample mean of a numerical variable is computed as the sum of all observations divided by the number of observations:
\begin{eqnarray}
\bar{x} = \frac{x_1 + x_2 + \cdots + x_n}{n}
\label{meanEquation}
\end{eqnarray}
where $x_1, x_2, \dots, x_n$ represent the $n$ observed values.}
\end{termBox}
\marginpar[\raggedright\vspace{-8mm}$n$\\\footnotesize sample size]{\raggedright\vspace{-8mm}$n$\\\footnotesize sample size}
\vspace{-2mm}

This arithmetic mean has an important property: if you know the mean weight of goods and the number of goods, then you immediately know the total weight. 
Other types of means are explored in Exercise~\ref{harmonic}. 
For example, if you know the geometric mean of the factors by which a stock price changes, and the number of such changes, then you know the overall change.

A \term{mode} corresponds to a prominent peak in a distribution. 
One precise definition is that the mode is the value with the highest frequency. 
However, real data sets often contain no repeated values, making this definition unhelpful.  
A more useful definition for continuous data is given in \Cref{probability}.

\label{variability}

We define the sample \term{variance}, denoted $s^2$, as
\begin{align*}
    s^2 = \frac{1}{n-1} \sum_{i=1}^n (x_i - \overline{x})^2 .
\end{align*}
\marginpar[\raggedright$s^2$\\\footnotesize sample variance]{\raggedright$s^2$\\\footnotesize sample variance}
We divide by $n-1$, rather than $n$, for reasons explained in \Cref{probability}. 
The \term{sample standard deviation} is $s = \sqrt{s^2}$.

Squaring deviations (rather than using fourth powers or absolute values) is deeply connected to the geometry of Euclidean space and Pythagoras’ Theorem. 
For independent random variables $X$ and $Y$, we have
\[
\Var(X + Y) = \Var(X) + \Var(Y),
\]
which parallels the relation $a^2 + b^2 = c^2$ for orthogonal side lengths.

The standard deviation is often more interpretable than the variance because it has the same units as the data. 
This also means that adding “one standard deviation” or “two standard deviations” to a measurement makes sense, whereas adding “one variance” does not.

Should we consider using $2\sigma$ instead of $\sigma$? 
Perhaps—but this is largely a matter of convention. 
For example, in the probability density function of the normal distribution, a factor of $\sqrt{2\pi}\sigma$ appears. 
Some argue that using $\tau = 2\pi$ instead of $\pi$ would simplify such formulas, but these choices are conventional.

\subsection{About those pie charts}
A \term{pie chart} is shown in \vref{emailNumberPieChart} alongside a bar plot. 
Although often criticized, pie charts are useful for quickly determining whether a proportion is close to 50\% or 25\%.

\begin{figure}[h]
   \centering
   \includegraphics[width=0.5\textwidth]{ch_intro_to_data/figures/pie-chart-s4cs}
   \caption{A pie chart for the data 1, 1, 1, 2, 1.}
   \label{emailNumberPieChart}
\end{figure}

\begin{exercise}
Could any shapes other than discs serve as useful “pie” charts?%
\footnote{For example, a hexagonal chart could help identify whether a proportion exceeds $1/6$.}
\end{exercise}

\section{Sampling}

This topic is treated in more detail in \emph{OpenIntro Statistics}. 
Here we offer a few remarks.

A \emph{simple random sample} is one in which each subset of $n$ observations is equally likely. 
For example, if we sample $n = 2$ individuals from the set $\{1,2,3\}$, the subsets $\{1,2\}$, $\{2,3\}$, and $\{1,3\}$ must each be equally likely. 
It is not enough that 1, 2, and 3 each appear with probability $1/3$.

If two samples are drawn from the same population distribution but the individuals receive different treatments, then differences in their outcomes can reasonably be attributed to the treatment—provided the difference is statistically significant. 
In observational studies, by contrast, there may always be unobserved factors. 
For example, suppose countries with many pianos have low malnutrition rates. 
It would be unreasonable to infer a causal relationship.

