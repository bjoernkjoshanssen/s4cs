\chapter*{Preface}

This book is a supplement to OpenIntro Statistics, which may be downloaded as a free PDF at \oiRedirect{textbook-openintro}{\color{black}\textbf{openintro.org}}. \vspace{3mm}


\noindent By choosing the title \emph{Statistics for Calculus Students} we intended to summarize the following prerequisite situation.\vspace{-1mm}
\begin{enumerate}
\setlength{\itemsep}{0mm}
\item[(1)] Students should have studied some calculus, say Calculus 1 and 2.
\item[(2)] Students could still be studying Calculus --- for instance, Calculus 3 (multivariable) is not assumed.
\item[(3)] Although it is essential for a full understanding of statistics, \emph{linear algebra} is not required to read this book.
\end{enumerate}
In particular, the book is designed to be appropriate for MATH 372 (Elementary Probability and Statistics) at University of Hawai\textquoteleft i at M\=anoa.

\noindent This is not a textbook on probability. While we review some basic concepts, we assume that students have some knowledge of probability as can be gained through Chapters 1--9 of Grinstead and Snell's \emph{Introduction to Probability}\footnote{Free download available at \url{http://www.dartmouth.edu/~chance/teaching_aids/books_articles/probability_book/amsbook.mac.pdf}.}.
Thus, we assume the student has seen some single-variable calculus-based probability, and some algebra-based statistics; and we intend to bridge the gap to single-variable calculus-based statistics.

\subsection*{Textbook overview}

The chapters of this book are as follows:
\begin{description}
\setlength{\itemsep}{0mm}
\item[1. Introduction to data.] Data structures, variables, summaries, graphics, and basic data collection techniques.
\item[2. Probability.] The basic principles of probability. %An understanding of this chapter is not required for the main content in Chapters~\ref{modeling}-\ref{multipleAndLogisticRegression}.
\item[3. Distributions of random variables.] Introduction to the normal model and other key distributions.
\item[4. Foundations for inference.] General ideas for statistical inference in the context of estimating the population mean.
\item[5. Inference for numerical data.] \hspace{1mm}Inference for one or two sample means using the \mbox{$t$-distribution}. %, and also comparisons of many means using ANOVA.
\item[6. Inference for categorical data.] Inference for proportions using the normal and $\chi^2$ distributions, as well as simulation and randomization techniques.
\item[7. Introduction to linear regression.] An introduction to regression with two variables. %Most of this chapter could be covered after Chapter~\ref{introductionToData}.
\item[8. Special topics.] Introduces maximum likelihood models and their applications to finding a hidden Markov model. Algebraic statistics, entropy, likelihood, Markov chains, and information criteria.
\end{description}

%\emph{OpenIntro Statistics} was written to allow flexibility in choosing and ordering course topics. The material is divided into two pieces: main text and special topics. The main text has been structured to bring statistical inference and modeling closer to the front of a course. Special topics, labeled in the table of contents and in section titles, may be added to a course as they arise naturally in the curriculum.


%\subsection*{Videos for sections and calculators}

%The \videohref[4mm]{textbook-openintro_videos} icon indicates that a section or topic has a video overview readily available. The~icons are hyperlinked in the textbook PDF, and the videos may also be found at
%\begin{center}
%\oiRedirect{textbook-openintro_videos}{\color{black}\textbf{www.openintro.org/stat/videos.php}}
%\end{center}


\subsection*{Examples, exercises, and appendices}

Examples and Guided Practice throughout the textbook may be identified by their distinctive bullets:

\begin{example}{Large filled bullets signal the start of an example.}
Full solutions to examples are provided and may include an accompanying table or figure.
 \end{example}

\begin{exercise}
Large empty bullets signal to readers that an exercise has been inserted into the text for additional practice and guidance. Students may find it useful to fill in the bullet after understanding or successfully completing the exercise. Solutions are provided for all Guided Practice in footnotes.\footnote{Full solutions are located down here in the footnote!}
\end{exercise}

There are exercises at the end of each chapter for practice or homework assignments. Odd-numbered exercise solutions are in Appendix~\ref{eoceSolutions}.

\subsection*{Google Sheets}
Occasional references to technology are made, in which cases our software of choice is \texttt{Google Sheets}. This choice was made on the basis of Google Sheets being freely available and only a few clicks away from University of Hawai\textquoteleft i students' Gmail accounts.

%\subsection*{OpenIntro, online resources, and getting involved}

%OpenIntro is an organization focused on developing free and affordable education materials. \emph{OpenIntro Statistics} is intended for introductory statistics courses at the college level. And this tome, \emph{Statistics for Calculus Students}, implies a college-level mathematics course.

%We encourage anyone learning or teaching statistics to visit \oiRedirect{textbook-openintro}{\color{black}\textbf{openintro.org}} and get involved. We also provide many free online resources, including free course software. 

%We value your feedback. If there is a particular component of the project you especially like or think needs improvement, we want to hear from you. You may find our contact information on the title page of this book or on the \oiRedirect{textbook-openintro_about}{About} section of \oiRedirect{textbook-openintro}{\color{black}\textbf{openintro.org}}.

\subsection*{Acknowledgements}

This project would not be possible without the passion and dedication of all those involved. %We hope you will join us in extending a \emph{thank you} to all the project volunteers.
The authors would like to thank faculty members in the Department of Mathematics at  University of Hawai\textquoteleft i at M\=anoa for their involvement and ongoing contributions. We~are also very grateful to the students who took our classes and provided us with valuable feedback over the last several years.

We are thankful to Outreach College at University of Hawai\textquoteleft i at M\=anoa for support under their Open Educational Resources grant program.

\subsection*{Second Edition}
This Second Edition is under development during Fall 2025 as part of the course MATH 372.
