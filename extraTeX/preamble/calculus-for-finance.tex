

MATH 252A: an opportunity to add more single-variable calculus (sequences, series, power laws) to S4CS, to be ready for the 2nd edition in Fall 2019.

% in conjunction with:

%Calculus 	ENGINEERING LEVEL
%David Guichard

%Active Calculus 	TOO WEIRD
%Matt Boelkins

%APEX Calculus ENGINEERING LEVEL
%Gregory Hartman, Brian Heinold, Troy Siemers, Dimplekumar Chalishajar

%Calculus in Context	DYNAMICAL SYSTEMS NSF PROJECT
%James Callahan, lead author

%Calculus I, II, III		ENGINEERING LEVEL
%Jerrold E. Marsden and Alan Weinstein

%Calculus			ENGINEERING LEVEL
%Gilbert Strang

%OpenStax Calculus	WEIRD XML THING
%Gilbert Strang and Edwin Herman, lead authors

%Vector Calculus
%Michael Corral


Danting Feng

She took my class MATH 372, Elementary Probability and Statistics, in the Spring semester of 2018.

Compared to the other students in the class, she had more advanced knowledge of how to conduct statistical tests using spreadsheets. She was able to thrive even if English is not her native language. She was also professional, friendly, and polite.

There were many students in the class so I was not able to get to know individual students well. I hope you appreciate that that is the reason for my response being somewhat brief.



=====

When teaching FIN 651 in Fall 2012, I noticed that even though the students were studying Shreve's emph{Stochastic Calculus for Finance},
they did not know \emph{Calculus}; hence this book.
Another goal of the book is to prepare students for Broverman's \emph{Mathematics of investment and credit} and for probability and statistics courses that use statistics. Thus we have a view toward the actuarial exams P and FM with the present text \emph{Calculus for Actuaries}.

We are proposing it as a textbook for MATH 203.
