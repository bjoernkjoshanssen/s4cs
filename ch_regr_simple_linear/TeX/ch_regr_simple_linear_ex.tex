\section{Exercises}

%__________________
%%\subsection{Line fitting, residuals, and correlation}

% 1

\eoce{\qt{Extreme correlations}
        Give specific numerical examples of observations with
        \begin{parts}
            \item 
                $r = 1$
                
            \item
                $r = 0$
                
            \item
                $r = -1$
        \end{parts}
        \footnote{By ``specific numerical'' is meant that, for instance, $\{(1,2),(0,-3),(0,0)\}$ could be a possible answer --- although it will be wrong as $r$ will not be exactly 0, 1 or -1 for this particular choice!}
}{}

\eoce{\qt{Cauchy-Schwarz}
        Let $u_1, u_2, \dots, u_n$ and $v_1, v_2, \dots, v_n$ be real numbers.
        \begin{parts}
            \item 
                Rearrange the terms in $(u_1 x + v_1)^2 + (u_2 x + v_2)^2 + \dots + (u_n x + v_n)^2$ to obtain a quadratic polynomial $a x^2+b x+c$ in $x$.
            \item
                Note that the polynomial in part a has at most one real root and therefore that its discriminant $b^2-4ac$ is at most zero. Calculate this discriminant to conclude that $\left(\sum_{i=1}^n u_i v_i \right)^2 \le \left(\sum_{i=1}^n u_i^2\right) \left(\sum_{i=1}^n v_i^2\right)$. This inequality (and its various generalizations) is known as the \textit{Cauchy-Schwarz inequality}.
        \end{parts}
}{}

\eoce{\qt{Theory of extreme correlations}
        Let $(x_1, y_1), (x_2, y_2), \dots (x_n, y_n)$ be observations.
        \begin{parts}
            \item 
                Show that 
                \begin{center}
                    $\displaystyle{\frac{1}{n-1} \sum_{i=1}^n \frac{x_i - \bar{x}}{s_x} \frac{y_i - \bar{y}}{s_y} = \frac{\sum_{i=1}^n(x_i - \bar{x})(y_i - \bar{y})}{\sqrt{\sum_{i=1}^n(x_i-\bar{x})^2} \sqrt{\sum_{i=1}^n(y_i-\bar{y})^2}}}$
                \end{center}
                Conclude that the term on the right is an equivalent formula for $r$.
                
            \item
                Apply the Cauchy-Schwarz inequality to $\left|\sum_{i=1}^n(x_i-\bar{x})(y_i-\bar{y}) \right|$ to conclude that $|r| \le 1$ and therefore that $-1 \le r \le 1$.
        \end{parts}
}{}

