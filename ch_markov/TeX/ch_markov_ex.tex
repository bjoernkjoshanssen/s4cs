\section{Exercises}

% 1
\eoce{\qt{Statistics algebra} 
	Let $X$ and $Y$ be Bernoulli random variables, both with parameter $p$, and let $p_{ij}$ with $i, j \in \left\{0, 1 \right\}$ be defined as in 8.1
	\begin{parts}
		\item 
			Explicitly calculate each value of $p_{ij}$ to directly verify the equalities listed in 8.1.
		\item
			Use part (a) to show that $p_{00}p_{01} - p_{01}^2 = 0$. Thus, a probability distribution of two Bernoulli variables can be expressed as a point in $\mathbb R^4$ that lies on the curve $xy - zw = 0, z=w$.
		\item
			Show that the converse of part b is true in some sense: that is,
            show that solutions of the curve $xy - zw = 0, z=w$ can be expressed as
            probability distributions of two Bernoulli variables by showing that they satisfy the equalities in part (a).
	\end{parts}
}{}

% 2
\eoce{\qt{Change of parameter for likelihood}
	Let $X$ be a binomial random variable with mean $\mu$. 
	\begin{parts}
		\item
			Derive a formula $L(\mu)$ (so $\mu$ is acting as a parameter, instead of a probability $p$ as in 8.3).
		\item
			Find the maximum likelihood of the formula in part a and explain why it might or might not be surprising.
	\end{parts}
}{}

% 3
\eoce{\qt{Another variant on the likelihood parameter}
	Same as the previous problem, but replace mean $\mu$ with variance $\sigma^2$.
}{}

% 4
\eoce{\qt{Akaike}
	Define the \textit{Akaike information criterion with correction} by
	\[
	AIC_C := AIC + \frac{2k^2+2k}{n-k-1}
	\]
	where $n$ is the number of samples and $k$ is the number of parameters.
	\begin{parts}
		\item
			Show that as the number of samples increases, the $AIC_C$ approaches the $AIC$.
		\item 
			Derive the formula for the $AIC_C$ of a linear regression model with $n$ samples.
		\item
			For which sample sizes is the $AIC_C$ a more accurate criterion than the BIC? 
	\end{parts}
}{}

% 5
\eoce{\qt{Equivalence of HMMS: alphabet}
	Show that any hidden Markov model is equivalent to one in which every state has exactly one transition for every symbol in the underlying language.
	(This is equivalent to the statement that any nondeterministic finite automaton is equivalent to a deterministic finite automaton.)
	Here, two hidden Markov models are equivalent if they accept the same symbols and output the same words with equal probabilities.
	Additionally, give an informal algorithm for finding such a hidden Markov model and give an upper bound on the number of states in the new Markov model.
}{}

% 6
\eoce{\qt{Functional margin}
	Consider a cluster of points $\vec{x}_i = (x_{1i}, x_{2i})$ separated by a line
	$\vec{w} \cdot \vec{x} - b = 0$ for some $b$ and $\vec{w} = (w_1, w_2)$.
	Let $y_i = 1$ if $\vec{x}_i$ is above this line and let $y_i = -1$ if $\vec{x}_i$ is below this line.
	Define the \textit{functional margin of $(w, b)$ with respect to $i$} by
	\begin{center}
		$\displaystyle{\gamma_i := y_i(w_1 x_1 + w_2 x_2 + b)}$
	\end{center}
	for $\vec{x} = (x_1, x_2)$. Fix $i$.
	\begin{parts}
		\item 
			Show that $\gamma_i$ is orthogonal (i.e. perpendicular) to the line $\vec{w} \cdot \vec{x} - b = 0$.
		\item
			Show that $\gamma_i(\vec{x}_i) > 0$.
	\end{parts}
}{}

% 7
\eoce{\qt{Geometric margin}
	Again consider $\vec{x}_i$, $\vec{w}$, $b$, and $y_i$ from the previous problem.
	Define the \textit{geometric margin of $(w, b)$ with respect to $i$} by
	\begin{center}
		$\displaystyle{\hat{\gamma}_i := y_i \left(\frac{w_1 x_1 + w_2 x_2 + b}{\sqrt{w_1^2 + w_2^2}} \right)}$
	\end{center}
	for $\vec{x} = (x_1, x_2)$. Again as in the previous problem, fix $i$.
	\begin{parts}
		\item 
			Show that the same properties of $\gamma_i$ in 6 also hold for $\hat{\gamma}_i$.
		\item
			Show that if $w_1, w_2$, and $b$ are replaced with $kw_1, kw_2$ and $kb$ for some constant $k$ in the formula for $\hat{\gamma_i}$,
			the resulting expression will still be equal to $\hat{\gamma}_i$.
			In other words, show that $\hat{\gamma}_i$ is \textit{invariant under rescaling of parameters}.
	\end{parts}
}{}

%8
\eoce{\qt{Logistic mean}
The derivative of the function used in logistic regression,
\[
F(x) = \frac1{1+e^{-x}}
\]
is a probability density function. Find it and its mean.

% Solution: $f(x)=f(-x)$ so by symmetry, if there is a mean then it is 0.
% And the fact there is one comes from the fact that $x/e^x$ is summable.
}{}

%9
\eoce{\qt{Logistic mode}
The derivative of the function used in logistic regression,
\[
F(x) = \frac1{1+e^{-x}}
\]
is a probability density function. Find it and its mode.
}{}

%10
\eoce{\qt{Logistic median}
The derivative of the function used in logistic regression,
\[
F(x) = \frac1{1+e^{-x}}
\]
is a probability density function. Find it and its median.
}{}

\eoce{\qt{Soccer regression}
For the soccer example in logistic regression, find the exact transfer fee for
which the probability of scoring a goal is increasing the most (per euro).
}{}

\eoce{\qt{Soccer regression II}
For the soccer example in logistic regression, find the exact transfer fee for
which the probability of scoring a goal becomes 50\%.
}{}

\eoce{\qt{Reflective logistic}\label{reflect}
Show that the logistic function
\[
F(x) = \frac1{1+e^{-x}}
\]
has the following properties:
\begin{enumerate}
	\item $\lim_{x\to \infty} F(x) = 1$;
	\item $\lim_{x\to -\infty} F(x) = 0$;
	\item $F(x)\ge 0$ for all $x$;
	\item $F(-x)=1-F(x)$.
\end{enumerate}
}{}

\eoce{\qt{Poor man's logistic}
Which of the properties of the logistic function in \ref{reflect}
does the function
\[
F(x) = \frac12\left(1 + \frac{x}{\sqrt{1+x^2}}\right)
\]
have?
}{}

\eoce{\qt{Sample exam: Uniform distribution on an interval}

Let \(X\) be uniformly distributed on the interval \([2,6]\).

\begin{enumerate}
\item[(a)] Find the probability density function \(f_X(x)\).
\item[(b)] Compute \(\mathbb{P}(3 \le X \le 5)\).
\item[(c)] Compute \(\mathbb{E}[X]\).
\end{enumerate}
}{}

\eoce{\qt{Fall 2025 EXAM QUESTION}
...
}{}

\eoce{\qt{Exponential distribution and integration}

A random variable \(T\) has density
\[
f_T(t) = \lambda e^{-\lambda t}, \qquad t \ge 0,
\]
with parameter \(\lambda = 2\).

\begin{enumerate}
\item[(a)] Compute the cumulative distribution function \(F_T(t)\).
\item[(b)] Find \(\mathbb{P}(T > 1)\).
\item[(c)] Compute \(\mathbb{E}[T]\) using an integral.
\end{enumerate}
}{}

\eoce{\qt{Fall 2025 EXAM QUESTION}
...
}{}
