\section{Exercises} \vspace{-2mm}

%__________________
%\subsection{Inference for a single proportion}

% 1

\eoce{\qt{One-half factorial\label{factorial}}
        Define the gamma function ${\displaystyle \Gamma(x) := \int_0^{\infty} t^{x-1} e^{-t}dt}$, where $x$ is a nonnegative real number.
        \begin{parts}
            \item 
                Show that $\Gamma(x+1) = x \Gamma(x)$ for all $x$. Conclude that $\Gamma(n+1) = n!$ for all natural numbers $n$. \textit{Hint: use integration by parts for the first part, and calculate $\Gamma(1)$ and apply induction for the second}.
            \item
                Calculate $\Gamma \left(\frac{1}{2} \right)$.
        \end{parts}
}{}

\eoce{\qt{Moments with $\chi^2$}
        Let a random variable $X$ have the \textit{chi-square density with $r$ degrees of freedom}, or \textit{$X$ is $\chi^2(r)$} for short, if its probability density function is of the form
        \begin{equation}\label{aeq}
        		f(x) = \frac{1}{\Gamma \left(\frac{r}{2} \right) 2^{\frac{r}{2}} }x^{\frac{r}{2}-1} e^{-\frac{x}{2}}
	\end{equation}
        \begin{parts}
            \item 
                Using Exercise \ref{factorial}(b), show that for any $r$, $f$ from (\ref{aeq}) is indeed a probability density function.
            \item
                If $X = \chi^2(r)$, show that $M_X(t) = \frac{1}{(1-2t)^{\frac{r}{2}}}$, where $M_X(t)$ is the moment generating function of $X$. 
            \item
                Use part (b) to show that if $X$ is a normal random variable with expected value 0 and variance 1, then $X^2$ is $\chi^2(1)$. 
            \item
                Use part (c) to show that if $X_1, \dots, X_n$ are mutually independent normal random variables, each with expected value 0 and variance 1, then $\sum_{i=1}^n X_i^2$ is $\chi^2(n)$.
        \end{parts}
}{}

\eoce{\qt{Difference of $\chi^2$}
        Let $X_1$ and $X_2$ be independent, let $X_1$ be $\chi^2(r_1)$, and let $X_1+X_2$ be $\chi^2(r)$ for some $r> r_1$. Show that $X_2$ is $\chi^2(r-r_1)$. 
}{}


\eoce{\qt{Gender and color preference\label{gender_color_preference_CI_concept}}
(This is Exercise 6.25 from \emph{OpenIntro Statistics}, copied here because of Exercise \ref{zain_email} below.)
A 2001 
study asked 1,924 male and 3,666 female undergraduate college students their favorite 
color. A 95\% confidence interval for the difference between the proportions of males 
and females whose favorite color is black $(p_{male} - p_{female})$ was calculated to 
be (0.02, 0.06). Based on this information, determine if the following statements are 
true or false, and explain your reasoning for each statement you identify as 
false.%\footfullcite{Ellis:2001}
\begin{parts}
\item We are 95\% confident that the true proportion of males whose favorite color is 
black is 2\% lower to 6\% higher than the true proportion of females whose favorite 
color is black.
\item We are 95\% confident that the true proportion of males whose favorite color is 
black is 2\% to 6\% higher than the true proportion of females whose favorite color is 
black.
\item 95\% of random samples will produce 95\% confidence intervals that include the 
true difference between the population proportions of males and females whose favorite 
color is black.
\item We can conclude that there is a significant difference between the proportions of 
males and females whose favorite color is black and that the difference between the two 
sample proportions is too large to plausibly be due to chance.
\item The 95\% confidence interval for $(p_{female} - p_{male})$ cannot be calculated 
with only the information given in this exercise.
\end{parts}
}{}

\eoce{\qt{Finding $\hat p_x$ and $\hat p_y$\label{zain_email}}
In Exercise \ref{gender_color_preference_CI_concept}, let $p_y$ be the male proportion and $p_x$ female.
We need the standard error of $\hat p_y-\hat p_x$ which will be
$$\sqrt{(\hat p_y(1-\hat p_y)/1924)+(\hat p_x(1-\hat p_x)/3666)}.$$
From this being 0.01 (since the width of the 95\%, hence $1.96\approx 2$ standard deviations, confidence interval is 0.02) and also $\hat p_y-\hat p_x=0.04$, deduce what they each are.
}{}

\eoce{\qt{American climate perspectives}
A 2018 survey\footnote{\url{https://ecoamerica.org/wp-content/uploads/2018/10/november-2018-american-climate-perspectives-survey.pdf}} found a value $\hat p=79\%$ and reported a margin of error of 3.5\%.
The sample size was $n=800$ and the confidence level was 95\% as usual. Did they calculate their standard error using $\hat p=.79$ or using an assumed value $p=1/2$? What would the margin of error be with the other choice? %Answer: $1.96\sqrt{.79(1-.79)/800}=2.8\%$ whereas $1.96\sqrt{.5(1-.5)/800}=3.5\%$.
}{}
