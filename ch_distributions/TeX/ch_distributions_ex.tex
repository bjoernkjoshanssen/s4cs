\section{Exercises}
% 1
\eoce{\qt{Skewed but real\label{skewed_but_real}} 
\begin{parts}
	\item 
		Give an example of real-life data that can be expressed as a normal distribution.
	\item
		Give an example of real-life data that can be expressed as a skewed normal distribution (Exercise \ref{skewed}).\footnote{
			Andel, J., Netuka, I. and Zvara, K. (1984) On threshold autoregressive processes. Kybernetika, 20, 89-106
		}
	\item
		Give an example of real-life data whose probability density function is constant-power function $f(x)=ax^{-k}$.
		Such a probability distribution is known as a \textit{power law distribution}.
\end{parts}
}{}

%2
\eoce{\qt{Three sigma} 
		Suppose you are flipping a coin.
\begin{parts}
	\item 
		What is the probability that you will see the first heads on the third flip? The tenth flip? The hundredth flip?
	\item
		What is the least $n$ such that seeing the first heads after $n$ flips results in a Z-score greater than 3?
		Use this answer to comment on the discussion on the first page of Chapter 2.
\end{parts}
}{}

%3
\eoce{\qt{Negative binomial\label{negative_binomial}}
	Consider a Bernoulli random variable with probability of success $p$.
	\begin{parts}
		\item 
			Find the probability that $k-1$ successes will be observed in $n-1$ trials.
		\item
			Use part (a) to find the probability that $k-1$ successes will be observed in $n-1$ trials and that the $n$th trial will be a success.
		\item
			Relate your answer to part (b) to the formula for the negative binomial distribution. Why or why not might this be surprising?
	\end{parts}
}{}

%4
\eoce{\qt{Tens}
	Let $X$ be normally distributed with mean $\mu$ and standard deviation $\sigma$.
	\begin{parts}
		\item 
			Assume $X$ takes on only positive values.
			Write an inequality that expresses that $n$ is at least $Z$ standard deviations greater than $\mu$, for some constants $\mu$ and $Z$.
		\item
			Rewrite the inequality in part a in terms of $Z^2$.
		\item
			Rewrite the inequality in part c assuming $X$ is a binomial variable, so that $\mu = np$ and $\sigma^2 = np(1-p)$ for some $p$.
			Using part a, intuitively describe why this assumption is reasonable.
		\item
			Assume that $np, np(1-p) > 10$. Conclude that $Z^2 < 10$.
		\item
			Calculate $\Phi(\sqrt{10}) - \Phi(-\sqrt{10})$ (where $\Phi(x)$ is defined in Chapter 1, Exercise 10).
			State why $np, np(1-p)$ are both required to be greater than 10 in 3.4.2.
	\end{parts}
}{}

% 5
%\eoce{\qt{GRE scores, Part II\label{GRE_cutoffs}} In Exercise~\ref{area_under_curve_1} we 
%saw...
%}{}
\eoce{\qt{Verifying the variance\label{verifying_the_variance}}
Complete the proof of Theorem \ref{varGeom} as follows.
\begin{parts}
\item Solve for $\beta$ to find $E(X^2)$.
\item Use $\sigma^2=E(X^2)-(E(X))^2$ to find $\sigma^2$.
\end{parts}
}{}
